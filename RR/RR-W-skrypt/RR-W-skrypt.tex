\documentclass[10pt,a4paper]{article}

% Pakiety
\usepackage[utf8]{inputenc} % Kodowanie znaków
\usepackage[T1]{fontenc} % Poprawne wyświetlanie polskich znaków
\usepackage[polish]{babel} % Ustawienie języka polskiego
\usepackage{geometry} % Ustawienia strony
\usepackage{fancyhdr} % Nagłówki i stopki
\usepackage{titlesec} % Formatowanie sekcji
\usepackage{amsmath} % Dodatkowe komendy matematyczne
\usepackage{enumitem}
\usepackage{amssymb} % Symbole matematyczne, w tym \mathbb
\usepackage{multicol}

% Ustawienia strony
\geometry{margin=2cm}

% Nagłówek
\pagestyle{fancy}
\fancyhf{} % Czyszczenie nagłówków i stopek
\renewcommand{\headrulewidth}{0pt} % Usunięcie linii w nagłówku


% Sekcja tytułowa na każdej stronie
\newcommand{\zagadnienie}[3]{%
    \clearpage % Każde zagadnienie na nowej stronie
    \noindent\textbf{#1} #2\\
    #3
}

% Dokument
\begin{document}

% treść wykładu
\zagadnienie{RR-W1, 07.10.25}{}
{
\subsubsection*{Wstęp}
\begin{quote}
$y'=y\cos x\quad y(x)=\int y(t)\cos x\; \mathrm{d}t \quad y=C\mathrm{e}^{\sin x}$\\
$y''+9y=0 \quad y(x)=C_1\sin 3x+C_2\cos 3x$\\\\
R.r. zwyczajne – szukana funkcja jednej zmiennej.\\
R.r. cząstkowe – szukana funkcja wielu zmiennych.\\\\
$\mathrm{np.}\quad u=u(x,y)\quad \frac{\partial u}{\partial x}+\frac{\partial u}{\partial y}=0 \quad \text{– równanie transportu}$\\
$u=u(t,x,y)\quad \frac{\partial u}{\partial t}-\left(\frac{\partial^2 u}{\partial x^2}+\frac{\partial^2 u}{\partial y^2}\right)=0 \quad \text{– równanie przewodnictwa ciepła}$\\
$u=u(t,x,y)\quad \frac{\partial^2 u}{\partial t^2}-\left(\frac{\partial^2 u}{\partial x^2}+\frac{\partial^2 u}{\partial y^2}\right)=0 \quad \text{– równanie falowe}$
\end{quote}

\subsubsection*{Definicja (R.r.z. rzędu $n \ge 1$)}
\begin{quote}
R.r.z rzędu $n \ge 1$, to zależność postaci
$F\left(y^{(n)}, y^{(n-1)}, \dots, y, t\right)=0$, gdzie $F$ jest daną\\
funkcją, a $y$ jest szukaną funkcją o
wartościach w $\mathbb{R}^m$, $m \ge 1$.
\end{quote}

\subsubsection*{Przykład}
\begin{quote}
$y(t)=\begin{pmatrix} y_1(t) \\ y_2(t) \end{pmatrix}$, $y'=Ay \quad A=\begin{pmatrix} 1 & -3 \\ 7 & 5 \end{pmatrix}$\\
$y(t)=\mathrm{e}^{At}\cdot C$
\end{quote}

\subsubsection*{Definicja (R.r.z. liniowe)}
\begin{quote}
R.r.z jest \textbf{liniowe}, jeżeli $F$ zależy liniowo
od $y^{(k)}_l$ dla $k=0, \dots, n$, $l=1, \dots, m$.\\
$\mathrm{np.}\quad y'=y^2$ \underline{nie jest liniowe}
\end{quote}

\subsubsection*{R.r. rzędu I (postać uwikłana)}
\begin{quote}
Uwikłane: $F(t, y, y') = 0$\\
Jawne: $y' = -\frac{F_t}{F_y}$
\end{quote}

\subsubsection*{R.r. rzędu II (ZDN)}
\begin{quote}
II ZDN: $\quad x''(t) = \frac{1}{m} F(t, x(t), x'(t)), \quad x(0)=x_0, \quad x'(0)=v_0$
\end{quote}

\subsubsection*{Przykład}
\begin{quote}
Sprawdzić, że $y^2+x^2=1$ jest rozwiązaniem $y y' + x = 0$\\
$$y^2(x)+x^2=1 \quad \mid \frac{\mathrm{d}}{\mathrm{d}x} \implies 2y(x)y'(x)+2x=0 \quad \mid \cdot \frac{1}{2} \quad  \implies y(x)y'(x)+x=0$$
\end{quote}

\subsubsection*{Definicja (Zagadnienie Cauchy'ego (Zagadnienie Początkowe))}
\begin{quote}
Zagadnienie Cauchy'ego (zagadnienie początkowe) to r.r. rzędu $n \ge 1$ z dodanym warunkiem początkowym:\\
$$y(t_0)=y_0$$
$$y'(t_0)=y_1$$
$$\dots$$
$$y^{(n-1)}(t_0)=y_{n-1}, \quad \text{gdzie } y_0, y_1, \dots, y_{n-1} \text{ dane}$$
\end{quote}

\subsubsection*{Uwaga}
\begin{quote}
Jeżeli mamy jedno równanie różniczkowe rzędu $n \ge 1$, to możemy je zapisać jako $n$ równań I rzędu.
\end{quote}

\subsubsection*{Przykład}
\begin{quote}
$y''' + (y'')^2 - \sin(y') + \sqrt{|y|} = \ln x$\\
$z_0(x) = y(x)$\\
$z_1(x) = y'(x)$\\
$z_2(x) = y''(x)$\\
$z'_0 = y' = z_1$\\
$z'_1 = y'' = z_2$\\
$z'_2 = y''' = -(z_2)^2 + \sin(z_1) - \sqrt{|z_0|} + \ln x$\\
$z = \begin{pmatrix} z_0 \\ z_1 \\ z_2 \end{pmatrix}$, to $z'=F(x,z)$\\
$$F(x,z) = \begin{pmatrix} z_1 \\ z_2 \\ -(z_2)^2 + \sin(z_1) - \sqrt{|z_0|} + \ln x \end{pmatrix}$$
\end{quote}

\subsubsection*{Definicja (R.r.z. rzędu I w postaci normalnej)}
\begin{quote}
R.r.z rzędu \textbf{I} w postaci \textbf{normalnej}, to równanie $y'=f(t,y)$.
\end{quote}

\subsubsection*{Uwaga}
\begin{quote}
Rozpatrzmy równanie różniczkowe:\\
$$(\star) \begin{cases} y'=f(t,y) \\ y(t_0)=y_0 \end{cases} \quad F: U \to \mathbb{R}^m \text{ dana, } U \subset \mathbb{R}^{m+1} \text{ otwarty,} \quad
(t_0, y_0) \in U, \quad y=\begin{pmatrix} y_1 \\ \vdots \\ y_m \end{pmatrix}.$$
Są tylko 2 twierdzenia o istnieniu rozwiązań $(\star)$.
\end{quote}

\subsubsection*{Twierdzenie Peano (o lokalnym istnieniu rozwiązań)}
\begin{quote}
Niech $Q=[t_0-a, t_0+a] \times \overline{B(y_0, b)}$, $a, b > 0$. Niech $f: Q \to \mathbb{R}^m$ ciągła i $M=\sup_a |f|$.\\
Wtedy istnieje rozwiązanie $(\star)$ określone
co najmniej na przedziale $(t_0-\alpha, t_0+\alpha)$,
gdzie $\alpha = \min \left\{a, \frac{b}{M}\right\}$.
\end{quote}

\subsubsection*{Twierdzenie Picarda-Lindelöfa (o lokalnym istnieniu i jednoznaczności)}
\begin{quote}
Przy założeniach tw. Peano, jeżeli dodatkowo $F$ jest
\textbf{lipschitzowsko ciągła względem $y$}, to \\ istnieje
\textbf{dokładnie jedno rozwiązanie} $(\star)$ określone na
przedziale $(t_0-\alpha, t_0+\alpha)$, gdzie $\alpha = \min \left\{a, \frac{b}{M}, \frac{1}{L}\right\}$.\\
$$\text{(Lipschitzowsko ciągła: } \exists L \ge 0 \quad \forall (t,y_1), (t,y_2) \in Q \quad \|F(t,y_1) - F(t,y_2)\| \le L \|y_1-y_2\| \text{)}$$
\end{quote}
}

\zagadnienie{RR-W2, 14.10.25}{}
{
    \subsubsection*{Przykład}
\begin{quote}
Równanie:\\
$y' = \sqrt{|y|}$ \\
$f(t,y) = \sqrt{|y|} \quad - \text{ciągła na } \mathbb{R}^2$ \\
Z tw. Peano istnieje rozwiązanie przechodzące przez dowolny ustalony punkt $(t_0, y_0) \in \mathbb{R}^2$. \\\\
\textbf{Co z jednoznacznością rozwiązań?}\\
Czy jeśli $Q$ - kostka, to czy istnieje $L \ge 0$ takie, że $\forall (t,y_1), (t,y_2) \in Q: |f(t,y_1) - f(t,y_2)| \le L|y_1 - y_2|$? \\
Rozpatrzmy przypadek $Q \subseteq \{(t,y): t \in \mathbb{R}, y > 0 \}$: \\
$$|\Delta| = |f(t,y_1) - f(t,y_2)| = |\sqrt{y_1} - \sqrt{y_2}| = \left| \frac{y_1 - y_2}{\sqrt{y_1} + \sqrt{y_2}} \right| = \frac{1}{|\sqrt{y_1} + \sqrt{y_2}|} \cdot |y_1 - y_2|$$
Niech $y_0 > 0$. Wtedy w pewnym otoczeniu $(y_0 - b, y_0 + b)$ mamy:\\
$\sqrt{y_1} + \sqrt{y_2} \ge \sqrt{y_0 - b} + \sqrt{y_0 - b} = 2\sqrt{y_0 - b} > 0$\\
stąd\\
$$\frac{1}{|\sqrt{y_1} + \sqrt{y_2}|} \le \frac{1}{2\sqrt{y_0 - b}} = L$$
W tym przypadku warunek Lipschitza jest spełniony lokalnie. \\\\
Rozpatrzmy przypadek $Q \subseteq \{(t,y): t \in \mathbb{R}, y < 0\}$\\
$$(\Delta) = |f(t,y_1) - f(t,y_2)| = |\sqrt{-y_1} - \sqrt{-y_2}| = \left| \frac{-y_1 + y_2}{\sqrt{-y_1} + \sqrt{-y_2}} \right| = \frac{1}{\sqrt{-y_1} + \sqrt{-y_2}} \cdot |y_1 - y_2| \leq L|y_1-y_2|$$\\
\textbf{Wniosek:}\\
Na zbiorze $\{(t,y): t \in \mathbb{R}, y \ne 0\}$ mamy \textbf{jednoznaczność} rozwiązań, tzn. $\forall (t_0, y_0) \in \mathbb{R}^2$, takiego że 
$y_0 \ne 0$ istnieje dokładnie jedno rozwiązanie przechodzące przez $(t_0, y_0)$ i określone na pewnym przedziale $(t_0 - \delta, t_0 + \delta)$ gdzie $\delta > 0$.\\

\textbf{Spróbujmy rozwiązać to równanie}\\
$y' = \sqrt{|y|}$\\\\
\textbf{Przypadek $y > 0$}\\
$y' = \sqrt{y} \quad$ (nie da się odcałkować) \\
$\frac{y'}{\sqrt{y}} = 1$\\
$\left( 2\sqrt{y} \right)' = 1$ \\
$\int 2\sqrt{y} = \int 1 dt$ \\
$2\sqrt{y} = t + C, \quad C \in \mathbb{R}$ \\
$\sqrt{y} = \frac{1}{2}(t + C), \quad C \in \mathbb{R}$ \\
$y = \frac{1}{4}(t + C)^2, \quad C \in \mathbb{R}$ \\
$t+C > 0 \quad \text{(bo y>0)}$\\
$\quad t > -C$

\textbf{Przypadek $y < 0$}\\
$y' = \sqrt{-y}$\\
$\frac{y'}{\sqrt{-y}} = 1$\\
$\left( -2\sqrt{-y} \right)' = 1$ \\
$-2\sqrt{-y} = t + C, \quad C \in \mathbb{R}$\\
$t + C < 0 \quad t < -C$\\
$\sqrt{-y} = -\frac{1}{2}(t + C)$ \\
$y = -\frac{1}{4}(t + C)^2$\\\\\\

\textbf{Przypadek $y = 0$}\\
Zauważmy, że $y \equiv 0$ też jest rozwiązaniem równania $y' = \sqrt{|y|}$.\\\\
\textbf{Powrót do warunku Lipschitza}\\
Czy jeżeli $Q = \{(t,y): |t - t_0| \le a, |y| \le b\}$ to czy na $Q$ zachodzi warunek Lipschitza?\\
Gdyby $|\sqrt{|y_1|} - \sqrt{|y_2|}| \le L|y_1 - y_2| \text{ na } Q$, to 
$y_2=0$ i $y_1=\frac{1}{n} \quad \sqrt{\frac{1}{n}} \leq L\frac{1}{n}$, więc $L$ nie istnieje.\\\\
\textbf{Uwaga!}\\
Jeśli $g$ ma ograniczoną pochodną na zbiorze $[a,b]$ to $g$ jest lipschitzowsko ciągła, bo z tw. Lagrange'a o wartości średniej $|g(x) - g(y)| = |g'(c)| \cdot |x - y| \le M \cdot |x - y|$
(gdzie $M = \sup_{x \in [a,b]} |g'(x)|$).\\

\textbf{Wniosek:} Nie jest spełniony warunek Lipschitza, więc spróbujmy pokazać niejednoznaczność rozwiązań w punktach $(t_0, 0)$ dla $t_0 \in \mathbb{R}$.\\

$$y_{t_0}(t) = \begin{cases} \frac{1}{4}(t - t_0)^2 & \text{dla } t \ge t_0 \\ -\frac{1}{4}(t - t_0)^2 & \text{dla } t < t_0 \end{cases}$$
$$\lim_{t \to t_0^+} \frac{1}{4}(t - t_0)^2 = 0 = \lim_{t \to t_0^-} -\frac{1}{4}(t - t_0)^2$$
Zadając $y_{t_0}(t_0) = 0$ otrzymujemy funkcję ciągłą.\\
$$y'_{t_0}(t) = \begin{cases} \frac{1}{2}(t - t_0) & \text{dla } t > t_0 \\ -\frac{1}{2}(t - t_0) & \text{dla } t < t_0 \end{cases}$$
$y_{t_0}$ jest klasy $C^1$ (ciągła i pochodna również ciągła).\\\\
Czy $y_{t_0}$ jest rozwiązaniem $y' = \sqrt{|y|}$? Tak.\\

Tutaj jest rysunek i z niego wniosek: Stąd przez $(t_0, 0)$ przechodzą przynajmniej 2 rozwiązania\\

Czy przez $(t_0, 0)$ przechodzi więcej rozwiązań?\\
Mamy \textbf{kontinuum różnych rozwiązań}.\\
Dla $\tilde{t}_0 < t_0$ określmy funkcję:\\
$$y_{\tilde{t}_0, t_0}(t) = \begin{cases} -\frac{1}{4}(t - \tilde{t}_0)^2 & \text{dla } t \le \tilde{t}_0 \\ 0 & \text{dla } \tilde{t}_0 \leq t \leq t_0 \\ \frac{1}{4}(t - t_0)^2 & \text{dla } t > t_0 \end{cases}$$
Łatwo sprawdzić, że $y_{\tilde{t}_0, t_0}$ jest klasy $C^1$ i spełnia $y' = \sqrt{|y|}$ z warunkiem początkowym $y(t_0) = 0$ i jest tych rozwiązań continuum.
\end{quote}

\subsubsection*{Definicja (Rozwiązanie osobliwe)}
\begin{quote}
Funkcja $y(t)$ będąca rozwiązaniem równania $y'=f(t,y)$ nazywa się \textbf{rozwiązaniem osobliwym}, jeżeli każdy punkt wykresu $y(t)$ jest punktem niejednoznaczności rozwiązań.\\
Przykład: $y' = \sqrt{|y|}$. Rozwiązanie $y \equiv 0$ jest rozwiązaniem osobliwym.
\end{quote}

\subsubsection*{Twierdzenie o całkowej postaci równania różniczkowego}
\begin{quote}
Niech $f$ ciągła. Wtedy $y \in C^1(I)$, gdzie $I$ jest przedziałem, spełnia zagadnienie Cauchy'ego:\\
$$\begin{cases} y' = f(t,y) \\ y(t_0) = y_0 \end{cases} \quad \iff \quad y \in C(I) \quad \text{i} \quad y(t) = y_0 + \int_{t_0}^{t} f(s,y(s)) ds \quad \text{dla } t \in I$$
\end{quote}

\subsubsection*{Dowód}
\begin{quote}
$\implies$\\
$y'=f(t,y) \quad | \int_{t_0}^t$\\
$y(t) - y(t_0) = \int_{t_0}^t f(s,y(s)) ds$\\\\
$\impliedby$ \\
$y(t) = y_0 + \int_{t_0}^t f(s,y(s)) ds \text{ jest klasy } C^1$\\
$\text{więc różniczkujemy}$
\end{quote}

\subsubsection*{Jak rozwiązać równanie całkowe?}
\begin{quote}
$$y(t) = y_0 + \int_{t_0}^{t} f(s,y(s)) ds \quad t \in (t_0 - \alpha, t_0 + \alpha) = I$$
gdzie $y(t)$ jest \textbf{niewiadomą funkcją}.\\\\
Niech $X = \{ w \in C(I) \mid w(t_0) = y_0 \}$. \\
Niech $P$ będzie określone na X wzorem $(Pw)(t) = y_0 + \int_{t_0}^{t} f(s,w(s)) ds$\\\\
Wtedy $P: X \to X$.\\
Wtedy $y$ jest rozwiązaniem r.r. $\begin{cases} y' = f(t,y) \\ y(t_s) = y_0 \end{cases} \iff y \text{ jest pkt. stałym } P \text{ tj. } y = Py$. \\\\
Przy odpowiednich założeniach można pokazać, że $P$ jest kontrakcją na pewnej podprzestrzeni $X$.\\
Zatem z tw. Banacha o punkcie stałym, punkt stały $y$ jest granicą ciągu iteracji $P$ tj.
$$y_{n+1} = P y_n \quad \text{i wtedy} \quad y_n \to y \quad \text{gdy } y = Py$$
Def. Ciąg $y_{n+1} = P y_n$ nazywamy \textbf{ciągiem przybliżeń Picarda}.
\end{quote}

\subsubsection*{Przykład}
\begin{quote}
\textbf{Wyznaczyć ciąg przybliżeń Picarda dla zagadnienia $\begin{cases} y' = y \\ y(0) = 1 \end{cases}$}\\
Postać całkowa:\\
$y(t) = 1 + \int_{0}^{t} y(s) ds$\\
$f(t,y) = y$\\
$y_0 = y(0) = 1$\\
Wzór iteracyjny: $y_{n+1}(t) = 1 + \int_{0}^{t} y_n(s) ds$\\\\
$y_0(t) = 1$\\
$y_1(t) = 1 + \int_{0}^{t} 1 ds = 1 + t$\\
$y_2(t) = 1 + \int_{0}^{t} (1 + s) ds = 1 + t + \frac{t^2}{2}$\\
$y_3(t) = 1 + \int_{0}^{t} \left( 1 + s + \frac{s^2}{2} \right) ds = 1 + t + \frac{t^2}{2} + \frac{t^3}{6}$\\
$\vdots$\\
$y_n(t) = \sum_{k=0}^{n} \frac{t^k}{k!} \quad $ hipoteza (przez indukcję można dowieść)\\\\
Wtedy $y_n \to y$ na $[-a, a] \; \forall_{a > 0}$\\\\
Zatem $y(t) = e^t$ (\emph{szereg zbiega do } $e^t$).
\end{quote}

}



\zagadnienie{RR-W3, 21.10.25}{}
{
    
}


\end{document}
