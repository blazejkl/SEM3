\documentclass[10pt,a4paper]{article}

% Pakiety
\usepackage[utf8]{inputenc} % Kodowanie znaków
\usepackage[T1]{fontenc} % Poprawne wyświetlanie polskich znaków
\usepackage[polish]{babel} % Ustawienie języka polskiego
\usepackage{geometry} % Ustawienia strony
\usepackage{fancyhdr} % Nagłówki i stopki
\usepackage{titlesec} % Formatowanie sekcji
\usepackage{amsmath} % Dodatkowe komendy matematyczne
\usepackage{enumitem}
\usepackage{amssymb} % Symbole matematyczne, w tym \mathbb
\usepackage{multicol}
\usepackage[most]{tcolorbox}
\tcbset{
  mybox/.style={
    colback=gray!5, % kolor tła
    colframe=black, % kolor ramki
    boxrule=0.5pt,  % grubość ramki
    arc=2pt,        % zaokrąglenie rogów
    left=6pt,right=6pt,top=6pt,bottom=6pt
  }
}

% Ustawienia strony
\geometry{margin=2cm}

% Nagłówek
\pagestyle{fancy}
\fancyhf{} % Czyszczenie nagłówków i stopek
\fancyhead[C]{RR-Ć-skrypt}
\fancyhead[R]{\thepage} % numer strony po prawej
\renewcommand{\headrulewidth}{0pt} % Usunięcie linii w nagłówku


% Sekcja tytułowa na każdej stronie
\newcommand{\zagadnienie}[3]{%
    \clearpage % Każde zagadnienie na nowej stronie
    \noindent\textbf{#1} #2\\
    #3
}

% Dokument
\begin{document}

% treść wykładu
\zagadnienie{RR-Ć1, 06.10.25}{}
{
    \\papierowe notatki, do uzupełnienia
}
\zagadnienie{RR-Ć2, 13.10.25}{}
{
    \begin{tcolorbox}[mybox]
    \subsubsection*{Twierdzenie o funkcji uwikłanej}
    \begin{quote}
    \end{quote}
    \end{tcolorbox}

    \subsubsection*{Z1.11a}
    \begin{quote}
    \end{quote}

    \subsubsection*{Z1.12a}
    \begin{quote}
    \end{quote}
        
    \subsubsection*{Z1.12b}
    \begin{quote}
    \end{quote}

    \subsubsection*{Z1.12d}
    \begin{quote}
    \end{quote}

    \begin{tcolorbox}[mybox]
    \subsubsection*{Definicja 1. (równanie o zmiennych rozdzielonych)}
    \begin{quote}
    \end{quote}

    \subsubsection*{Twierdzenie 2.}
    \begin{quote}
    \end{quote}
    \end{tcolorbox}

    \subsubsection*{Z2.1a}
    \begin{quote}
    \end{quote}

    \subsubsection*{Z2.1c}
    \begin{quote}
    \end{quote}

    \subsubsection*{Z2.1b}
    \begin{quote}
    \end{quote}

    \subsubsection*{Z2.2}
    \begin{quote}
    \end{quote}

    \begin{tcolorbox}[mybox]
    \subsubsection*{Definicja 3. (równanie jednorodne)}
    \begin{quote}
    \end{quote}
    \end{tcolorbox}

    \subsubsection*{Z2.4a}
    \begin{quote}
    \end{quote}

    \subsubsection*{Z2.5}
    \begin{quote}
    \end{quote}
}
\zagadnienie{RR-Ć3, 20.10.25}{}
{
    \subsubsection*{Z2.6b}
    \begin{quote}
    \end{quote}

    \subsubsection*{Z2.6c}
    \begin{quote}
    \end{quote}

    \subsubsection*{Z2.6d}
    \begin{quote}
    \end{quote}

    \subsubsection*{Z2.6e}
    \begin{quote}
    \end{quote}

    \begin{tcolorbox}[mybox]
    \subsubsection*{Definicja 1. (potecjał pola wektorowego)}
    \begin{quote}
    \end{quote}

    \subsubsection*{Twierdzenie 2.}
    \begin{quote}
    \end{quote}
    \end{tcolorbox}

    \subsubsection*{Z3.1a}
    \begin{quote}
    \end{quote}
        
    \subsubsection*{Z3.1c}
    \begin{quote}
    \end{quote}

    \subsubsection*{Z3.1d}
    \begin{quote}
    \end{quote}

    \subsubsection*{Z3.2}
    \begin{quote}
    \end{quote}

    \begin{tcolorbox}[mybox]
    \subsubsection*{Definicja 3. (czynnik całkujący)}
    \begin{quote}
    \end{quote}
    \end{tcolorbox}

    \subsubsection*{Z3.3a}
    \begin{quote}
    \end{quote}

    \subsubsection*{Z3.3c}
    \begin{quote}
    \end{quote}

    \begin{tcolorbox}[mybox]
    \subsubsection*{Definicja 4.}
    \begin{quote}
    \end{quote}
    \end{tcolorbox}
}
\zagadnienie{prep do RR-Ć4, 26.10.25}{}
{
    \subsubsection*{Z3.4}
    \begin{quote}
    \end{quote}

    \subsubsection*{Z3.5}
    \begin{quote}
    \end{quote}

    \subsubsection*{Z3.6}
    \begin{quote}
    \end{quote}
        
    \subsubsection*{Z4.1}
    \begin{quote}
    \end{quote}

    \subsubsection*{Z4.2}
    \begin{quote}
    \end{quote}

    \subsubsection*{Z4.3}
    \begin{quote}
    \end{quote}
}
\zagadnienie{RR-Ć4, 27.10.25}{}
{
}

\end{document}
