\documentclass[10pt,a4paper]{article}

% Pakiety
\usepackage[utf8]{inputenc} % Kodowanie znaków
\usepackage[T1]{fontenc} % Poprawne wyświetlanie polskich znaków
\usepackage[polish]{babel} % Ustawienie języka polskiego
\usepackage{geometry} % Ustawienia strony
\usepackage{fancyhdr} % Nagłówki i stopki
\usepackage{titlesec} % Formatowanie sekcji
\usepackage{amsmath} % Dodatkowe komendy matematyczne
\usepackage{enumitem}
\usepackage{amssymb} % Symbole matematyczne, w tym \mathbb
\usepackage{multicol}

% Ustawienia strony
\geometry{margin=2cm}

% Nagłówek
\pagestyle{fancy}
\fancyhf{} % Czyszczenie nagłówków i stopek
\renewcommand{\headrulewidth}{0pt} % Usunięcie linii w nagłówku


% Sekcja tytułowa na każdej stronie
\newcommand{\zagadnienie}[3]{%
    \clearpage % Każde zagadnienie na nowej stronie
    \noindent\textbf{#1} #2\\
    #3
}

% Dokument
\begin{document}

% treść wykładu
\zagadnienie{AM3-W1, 08.10.25 (niesformatowane z wykładem) (bez uzupełnionych dowodów)}{}
{
    \subsubsection*{Definicja (przestrzeni topologicznej).}
    \begin{quote}
    \textbf{Przestrzenią topologiczną} nazywamy parę $(X, \mathcal{T})$, gdzie $X$ to zbiór, a $\mathcal{T} \subset 2^X$ oraz spełnione są warunki:\\
    (1) $\emptyset, X \in \mathcal{T}$,\\
    (2) $\forall_{\alpha \in A} U_{\alpha} \in \mathcal{T} \Rightarrow \bigcup \{U_{\alpha}: \alpha \in A\} \in \mathcal{T}$,\\
    (3) $U_1, U_2 \in \mathcal{T} \Rightarrow U_1 \cap U_2 \in \mathcal{T}$.\\\\
    Zbiory należące do topologii nazywamy \textbf{otwartymi}. \\
    Dopełnienia zbiorów otwartych nazywamy zbiorami \textbf{domkniętymi}. \\\\
    $\mathcal{T} = 2^X \leftarrow$ \textbf{topologia dyskretna} \\
    $\mathcal{T} = \{\emptyset, X\} \leftarrow$ \textbf{topologia antydyskretna}
    \end{quote}


    \subsubsection*{Definicja (przestrzeni metrycznej).}
    \begin{quote}
    \textbf{Przestrzenią metryczną} nazywamy parę $(X, d)$, gdzie $X$ to zbiór, a $d: X \times X \to [0, \infty)$ to funkcja, zwana \textbf{metryką}, spełniająca następujące warunki:\\
    (1) $d(x, y) = d(y, x)$,\\
    (2) $d(x, y) = 0 \iff x = y$,\\
    (3) $d(x, y) \leq d(x, z) + d(z, y)$.\\
    Trzecia własność jest równoważna \textbf{dolnej nierówności trójkąta}:
    $$
    d(x, z) \geq |d(x, y) - d(y, z)|.
    $$
    \end{quote}

    \subsubsection*{Definicja (topologii zadanej przez metrykę, kuli).}
    \begin{quote}
    Rodzinę $\mathcal{T}_d$ nazywamy \textbf{topologią zadaną przez metrykę $\mathbf{d}$}, jeśli zachodzi:
    $$U \in \mathcal{T}_d \iff \forall_{x \in U} \exists_{\varepsilon > 0} B(x, \varepsilon) \subset U,$$
    gdzie
    $$B(x, \varepsilon) = \{y \in X : d(x, y) < \varepsilon\}.$$
    \end{quote}

    \subsubsection*{Twierdzenie (o topologii zadanej przez metrykę).}
    \begin{quote}
    Rodzina $\mathcal{T}_d$ jest \emph{topologią}.
    \end{quote}

    \subsubsection*{Definicja (topologii metryzowalnej).}
    \begin{quote}
    Jeśli możemy znaleźć metrykę zadającą topologię, to nazywamy tę topologię \textbf{metryzowalną}.
    \end{quote}

    \subsubsection*{Definicja (wnętrza zbioru).}
    \begin{quote}
        Wnętrzem zbioru $A$ nazwiemy sumę wszystkich zbiorów otwartych zawartych w $A$:
        $$\mathrm{int}(A) = \bigcup \{U \in \mathcal{T} : U \subset A\}$$
    \end{quote}

    \subsubsection*{Definicja (domknięcia zbioru).}
    \begin{quote}
    \textbf{Domknięciem} $A$ nazywamy przecięcie wszystkich zbiorów domkniętych zawierających $A$:
    $$\overline{A} = \bigcap \{F : F^c \in \mathcal{T}, F \supset A\}$$
    \end{quote}

    \subsubsection*{Stwierdzenie (o wnętrzu i domknięciu).}
    \begin{quote}
    \begin{enumerate}[label=\alph*)]
        \item $\mathrm{int}\, A \subset A \subset \overline{A}$
        \item $\mathrm{int}\, A \in \mathcal{T}, (\overline{A})^c \in \mathcal{T}$
        \item $\mathrm{int}\, A \supset \{x \in A: \exists_{x \in U \in \mathcal{T}} \; U \subset A\}$
        \item $\overline{A} = (\mathrm{int}\, A^c)^c$
    \end{enumerate}
    \end{quote}

    \subsubsection*{Wniosek}
    \begin{quote}
    $x \in \overline{A}^c \iff \exists_{U \in \mathcal{T}} \; x \in U \subset A^c \quad (\text{wynika z d) i c) stw., bo } \exists_{U \in \mathcal{T}} \; x \in U \subset A^c \iff x \in \mathrm{int}\, A^c)$
    \end{quote}

    \subsubsection*{Definicja (Brzegu).}
    \begin{quote}
    Brzegiem zbioru $A$ nazywamy zbiór $\partial A = \overline{A} \setminus \mathrm{int}\, A$
    \end{quote}

    \subsubsection*{Wniosek}
    \begin{quote}
        $x \in \partial A \iff \forall_{x \in U \in \mathcal{T}} \; U \cap A \neq \emptyset, U \cap A^c \neq \emptyset$
    \end{quote}

    \subsubsection*{Twierdzenie (o trójpodziale).}
    \begin{quote} Dla każdego zbioru $A$ przestrzeń $X$ rozpada się na sumę trzech rozłącznych zbiorów: \\
    $$X = \mathrm{int}(A) \sqcup \mathrm{int}(A^c) \sqcup \partial A.$$
    \end{quote}

    \subsubsection*{Przykład}
    \begin{quote}
    $\mathrm{int}\, \mathbb{Q} = \emptyset, \mathrm{int}\, \mathbb{Q}^c = \emptyset, \mathrm{int}\, \partial \mathbb{Q} = \mathbb{R}$
    \end{quote}

    \subsubsection*{Definicja (zbiór gęsty).}
    \begin{quote}
    $A$ jest gęsty w $X \iff \overline{A} = X$
    \end{quote}

    \subsubsection*{Przykład (Zbiór Cantora).}
    \begin{quote}
    Zbiór Cantora: domknięty, nigdzie gęsty, puste wnętrze (konstr. geom: zaczynamy od $C_0 = \langle 0, 1 \rangle$, z $C_0$ usuwamy środkową $\frac{1}{3}$ i otrzymujemy $C_1$, z każdego odcinka $C_1$ robimy to samo, $\dots$, powtarzamy tę procedurę w nieskończoność, zbiór Cantora to zbiór punktów, które pozostaną po wykonaniu $\infty$ wielu kroków, formalnie $C = \bigcap_{n=0}^{\infty} C_n$
    \end{quote}

    \subsubsection*{Definicja (zbioru granicznego).}
    \begin{quote}
    Niech $(X, \mathcal{T})$ będzie przestrzenią topologiczną, a $(x_n) \in X$. Mówimy, że $x_0$ należy do \textbf{zbioru granicznego} ciągu $(x_n)$, jeżeli \\
    $$\forall_{U \in \mathcal{T}, x_0 \in U} \; \#\{n \in \mathbb{N}: x_n \notin U\} < \infty.$$
    Zbiór graniczny oznaczamy $\Lambda(x_n)$.
    \end{quote}

    \subsubsection*{Definicja (przeliczalnej bazy otoczeń $x_0$).}
    \begin{quote}
    Mówimy, że $x_0$ ma przeliczalną bazę otoczeń $\iff$ $\exists_{\{V_n\}_{n=1}^{\infty} \subset \mathcal{T}} \; \forall_{n} \; x_0 \in x_n, \forall_{x \in U \in T} \; \exists_n \; V_n \subset U$
    \end{quote}

    \subsubsection*{Definicja (przeliczalna baza otoczeń p. topologicznej).}
    \begin{quote}
    Mówimy, że przestrzeń topologiczna $(X, \mathcal{T})$ ma przeliczalną bazę otoczeń (\textbf{pierwszy warunek przeliczalności}) $\iff$ każdy punkt $x \in X$ ma przeliczalną bazę otoczeń.
    \end{quote}

    \subsubsection*{Uwaga}
    \begin{quote}
    Przestrzenie metryczne mają przeliczalne bazy otoczeń.
    \end{quote}

    \subsubsection*{Przykłady}
    \begin{quote}
    1. $\mathcal{T} = 2^X$ \\
    $x_k \in \Lambda(x_n) \iff \exists_{n_0 \in \mathbb{N}} \; \forall_{n \geq n_0} \; x_n = x_0$ \\
    2. $\mathcal{T} = \{\emptyset, X\}$ \\
    $\Lambda(x_n) = X$
    \end{quote}

    \subsubsection*{Definicja (topologii Hausdorffa).}
    \begin{quote}
    Parę $(X, \mathcal{T})$ nazywamy \textbf{topologią Hausdorffa}, jeżeli spełniony jest warunek \\
    $$\forall_{x_1, x_2 \in X, x_1 \neq x_2} \; \exists_{U_1 \ni x_1, U_2 \ni x_2, U_1, U_2 \in \mathcal{T}} \; U_1 \cap U_2 = \emptyset.$$
    \end{quote}

    \subsubsection*{Uwaga.}
    \begin{quote}
    Dla dowolnego ciągu $(x_n)$ w topologii Hausdorffa zachodzi $\#\Lambda(x_n) \leq 1$.
    \end{quote}

    \subsubsection*{Uwaga.}
    \begin{quote}
    Dowolna przestrzeń metryczna jest przestrzenią Hausdorffa.
    \end{quote}

}

\zagadnienie{AM3-W2, 08.10.25 (niesformatowane z wykładem) (bez uzupełnionych dowodów)}{}
{
    \subsubsection*{Definicja (domknięcia ciągowego).}
    \begin{quote}
    Domknięciem ciągowym $A$ nazywamy zbiór \\
    $$\mathrm{scl}(A) := \{x_0 \in X: \exists_{(x_n) \in A} \; x_0 \in \Lambda((x_n)_{n=1}^{\infty})\}.$$
    \end{quote}

    \subsubsection*{Twierdzenie (o domknięciu i domknięciu ciągowym).}
    \begin{quote}
    Zachodzi zawieranie $\mathrm{scl}(A) \subset \overline{A}$, a jeżeli topologia jest metryzowalna, to $\mathrm{scl}(A) = \overline{A}$.
    \end{quote}

    \subsubsection*{Uwaga.}
    \begin{quote}
    Bez założenia metryzowalności równość nie jest prawdziwa.
    \end{quote}

    \subsubsection*{Wniosek}
    \begin{quote}
    Kryterium domkniętości zbioru w przestrzeni metrycznej: $A = \mathrm{scl}(A)$
    \end{quote}

    \subsubsection*{Definicja (funkcji ciągłej).}
    \begin{quote}
    Funkcję $f : (X, \mathcal{T}_X) \to (Y, \mathcal{T}_Y)$ nazywamy \textbf{ciągłą}, jeżeli przeciwobraz dowolnego zbioru otwartego jest otwarty, tzn. \\
    $$\forall_{U \in \mathcal{T}_Y} \; f^{-1}(U) \in \mathcal{T}_X.$$
    \end{quote}

    \subsubsection*{Definicja (funkcji ciągłej w sensie Heinego).}
    \begin{quote}
    Funkcję $f : (X, \mathcal{T}_X) \to (Y, \mathcal{T}_Y)$ nazywamy \textbf{ciągłą w sensie Heinego}, jeżeli \\
    $$\forall_{(x_n) \in X} \; x_0 \in \Lambda((x_n)_{n=1}^{\infty}) \implies f(x_0) \in \Lambda((f(x_n))_{n=1}^{\infty}),$$
    gdzie w przestrzeniach Hausdorffa zbiór graniczny zamieniany jest na granicę.
    \end{quote}

    \subsubsection*{Twierdzenie (o równoważnych warunkach ciągłości).}
    \begin{quote}
    Jeśli $f$ jest ciągła, to zachodzi warunek Heinego ciągłości. Jeśli topologia jest metryzowalna, to implikacja zachodzi też w drugą stronę.
    \end{quote}

    \subsubsection*{Definicja (bazy topologii).}
    \begin{quote}
    Bazą topologii $\mathcal{T}$ nazywamy taką rodzinę $\mathcal{B} \subset \mathcal{T}$, że dowolny zbiór $U \in \mathcal{T}$ ma postać \\
    $$U = \bigcup \{V \in \mathcal{B}' : \mathcal{B}' \subset \mathcal{B}\}.$$
    \end{quote}

    \subsubsection*{Stwierdzenie (o bazie topologii $\mathcal{T}_d$).}
    \begin{quote}
    Kule otwarte stanowią bazę w $\mathcal{T}_d$.
    \end{quote}

    \subsubsection*{Definicja (podbazy topologii).}
    \begin{quote}
    Rodzinę $\mathcal{P}$ nazwiemy \textbf{podbazą} topologii $\mathcal{T}$, jeżeli zbiór $\mathcal{B}$ taki, że \\
    $$\mathcal{B} = \left\{\bigcap \{V \in \mathcal{P}' : \mathcal{P}' \subset \mathcal{P}, \#\mathcal{P}' < \infty\}\right\}$$
    jest bazą w $\mathcal{T}$.
    \end{quote}

    \subsubsection*{Uwaga.}
    \begin{quote}
    Zbiór $\mathcal{B}$ zawiera całą przestrzeń $X$ jako przecięcie pustej rodziny dla $\mathcal{P}' = \emptyset$. Przestrzeń ta jest również zamknięta ze względu na przecięcie dwóch zbiorów z $\mathcal{B}$. Wynika stąd, że każda rodzina $\mathcal{P} \subset 2^X$ określa pewną topologię na $X$.
    \end{quote}

    \subsubsection*{Sposoby zadawania topologii}
    \begin{quote}
    \textbf{Przez bazę}: \\
    $\mathcal{B} \subset 2^X$. Definiujemy $\mathcal{T}(\mathcal{B}) := \{\bigcup \{V \in \mathcal{B}_0 : \mathcal{B}_0 \subset \mathcal{B}\}\}$
    \end{quote}

    \subsubsection*{Przykłady}
    \begin{quote}
    W przestrzeni metrycznej $\mathcal{T}(\{B(x, r) : x \in X, r > 0\}) = \mathcal{T}$. \\
    $\mathcal{B} = \{[a, b] : a < b, a, b \in \mathbb{R}\}$. \\
    $(0, 1) = \bigcup \{[\frac{1}{n}, 1 - \frac{1}{n}] : n \in \mathbb{N}\}$. \\
    Topologia nie spełnia c), bo $[a, b] \cap [b, c] = \{b\}$.
    \end{quote}

    \subsubsection*{Fakt}
    \begin{quote}
    $\mathcal{T}(\mathcal{B})$ jest topologią $\iff$ \\
    a) $\bigcup_{U \in \mathcal{B}} U = X$. \\
    b) $\forall_{U, V \in \mathcal{B}} \; U \cap V \in \mathcal{T}(\mathcal{B})$
    \end{quote}

    \subsubsection*{Definicja (iloczynu kartezjańskiego).}
    \begin{quote}
    Iloczyn kartezjański definiujemy jako \\
    $$\prod_{\alpha \in A} X_{\alpha} := \{\varphi : A \to \bigcup_{\alpha \in A} X_{\alpha} : \forall_{\alpha \in A} \; \varphi(\alpha) \in X_{\alpha}\}.$$
    \end{quote}

    \subsubsection*{Pewnik wyboru}
    \begin{quote}
    $$\forall_{\alpha \in A} \; X_{\alpha} \neq \emptyset \implies \prod_{\alpha \in A} \{X_{\alpha} : \alpha \in A\} \neq \emptyset$$
    $\exists_{\varphi} \; \varphi : A \to \bigcup_{\alpha \in A} X_{\alpha}, \; \forall_{\alpha \in A} \; \varphi(\alpha) \in X_{\alpha}$ (\textbf{selektor})
    \end{quote}
}

\zagadnienie{AM3-W3, 15.10.25 (bez uzupełnionych dowodów)}{}
{

    \begin{quote}
    $(X_\alpha, \mathfrak{T}_\alpha)$, $X = \prod \{X_\alpha : \alpha \in A\}$ \\
    \end{quote}

    \subsubsection*{Definicja cylindra otwartego}
    \begin{quote}
    Cylindrem (otwartym) w $X$ nazywamy zbiór: \\
    $C_\alpha(U_\alpha) = \{\varphi \in X : \varphi_\alpha \in U_\alpha\}$, $U_\alpha \in \mathfrak{T}_\alpha$
    \end{quote}

    \subsubsection*{Definicja Topologii Tichonowa (produktowej)}
    \begin{quote}
    Topologia Tichonowa na $X$ jest zadana przez podbazę \\
    $\mathcal{P}_X = \{C_\alpha(U_\alpha) : \alpha \in A, U_\alpha \in \mathfrak{T}_\alpha\}$ \\
    $\mathcal{B}_X = \mathcal{B}(\mathcal{P}_X) = \{\bigcap_{j=1}^n C_{\alpha_j}(U_{\alpha_j}) : n=0,1,2,\dots, \alpha_i \in A, U_{\alpha_i} \in \mathfrak{T}_{\alpha_i}\}$ \\
    Dla produktu skończenie wielu przestrzeni $X_1, \dots, X_n$, \\
    $\mathcal{B}_X = \{\prod_{j=1}^n U_j : \forall_j \; U_j \in \mathfrak{T}_j\}$
    \end{quote}
    }

    \subsubsection*{Stwierdzenie o zbieżności w topologii Tichonowa}
    \begin{quote}
    Jeśli dany jest ciąg $(\varphi_n)_{n=0}^\infty$ w $X$, to $\varphi_0 \in \mathrm{Lim} (\varphi_n)_{n=1}^\infty \iff \forall_{\alpha \in A} \; \varphi_0(\alpha) \in \mathrm{Lim} (\varphi_k(\alpha))_{k=1}^\infty$ \\
    $(X_1, d_1), \dots, (X_n, d_n)$ - prz. metryczne , gdzie $X = \prod_{i=1}^n X_i$, $\mathfrak{T}$ - top. Tichonowa
    \end{quote}

    \subsubsection*{Definicja metryki produktowej}
    \begin{quote}
    $d_\pi((x_i)_{i=1}^{\hat{n}}, (\hat{x}_i)_{i=1}^{\hat{n}}) = \max \{d_i(x_i, \hat{x}_i) : i=1, \dots, \hat{n}\}$ \\
    \end{quote}

    \subsubsection*{Tw. o równości topologii dla skończonego produktu}
    \begin{quote}
    $\mathfrak{T}_X = \mathfrak{T}_{d_\pi}$, $x_0 = (x_1, \dots, x_{\hat{n}})$
    \end{quote}

    \subsubsection*{Definicja pokrycia i podpokrycia}
    \begin{quote}
    $(X, \mathfrak{T})$, $Z \subset X$. Podrodzina zbiorów otwartych $\{U_\alpha\}_{\alpha \in A}$ w $X$ \\
    nazywa się \textbf{pokryciem (otwartym)} zbioru $Z \iff Z \subset \bigcup \{U_\alpha : \alpha \in A\}$. \\
    Powiemu, że $\{U_\alpha\}_{\alpha \in \hat{A}}$ jest \textbf{podpokryciem} $\{U_\alpha\}_{\alpha \in A} \iff \hat{A} \subset A$ i jest pokryciem
    \end{quote}

    \subsubsection*{Definicja zbioru zwartego}
    \begin{quote}
    Zbiór $K \subset X$ nazywamy \textbf{zwartym} $\iff$ dla każdego pokrycia otwartego \\
    $K$ istnieje jego podpokrycie skończone. 
    \end{quote}

    \subsubsection*{Twierdzenie o zbiorach zwartych i domkniętych}
    \begin{quote}
    $K \subset X$, $(X, \mathfrak{T})$ \\
    a) Jeśli $X$ jest zwarta, a $K$ domknięty, to $K$ jest zwarty \\
    b) Jeśli $X$ jest Hausdorffa i $K$ jest zwarty, to $K$ jest domknięty
    \end{quote}

\zagadnienie{AM3-W4, 15.10.25 (bez uzupełnionych dowodów)}{}
{
    \subsubsection*{Twierdzenie Tichonowa}
    \begin{quote}
    Jeśli $\forall_{\alpha \in A} \; (X_\alpha, \mathfrak{T}_\alpha)$ jest zwarta, to $\prod \{X_i : i \in A\}$ jest zwarty.
    \end{quote}

    \subsubsection*{Twierdzenie o funkcjach ciągłych na zbiorach zwartych}
    \begin{quote}
    $f: X \to Y$ ciągła, $K \subset X$ jest zwarty. \\
    Wówczas: \\
    a) $f(K)$ jest zwarty w $Y$. \\
    b) Jeśli $Y = \mathbb{R}$, to $f$ przyjmuje swoje kresy na $K$: \\
    $\exists_{x_{\mathrm{min}} \in K} \; f(x_{\mathrm{min}}) = \inf \{f(x) : x \in K\}$ \\
    $\exists_{x_{\mathrm{max}} \in K} \; f(x_{\mathrm{max}}) = \sup \{f(x) : x \in K\}$ \\
    c) Jeśli $(X, d_X)$ oraz $(Y, d_Y)$ są prz. met. , to $f$ jest jednostajnie ciągła na $K$: \\
    $\forall_{\varepsilon > 0} \; \exists_{\delta > 0} \; \forall_{x, y \in K} \; d_X(x, y) < \delta \implies d_Y(f(x), f(y)) < \varepsilon$.
    \end{quote}

    $(X, d)$, $Z \subset X$, ustalmy $\varepsilon > 0$. \\
    \subsubsection*{Definicja sieci}
    \begin{quote}
    Zbiór skończony $\{x_1, \dots, x_p\} \subset X$ jest \textbf{siecią o prześwicie $\varepsilon$} dla \\
    $Z \iff \bigcup_{j=1}^p B(x_j, \varepsilon) \supset Z$.
    \end{quote}

    \subsubsection*{Definicja zbioru całkowicie ograniczonego}
    \begin{quote}
    Zbiór $Z \subset X$ nazywamy \textbf{całkowicie ograniczonym} $\iff$ \\
    $\iff \forall_{\varepsilon > 0}$ istnieje sieć o prześwicie $\varepsilon$.
    \end{quote}

    \subsubsection*{Uwaga}
    \begin{quote}
    Jeśli istnieje sieć o prześwicie $\varepsilon$, to istnieje sieć o prześwicie \\
    $2\varepsilon$ taka, że $\forall_{i} \; x_i \in Z$.
    \end{quote}

    \subsubsection*{Definicja ciągu zdystansowanego}
    \begin{quote}
    Ciąg $(x_n)$ nazywamy \textbf{zdystansowanym o pewne $\varepsilon > 0$} $\iff$ \\
    $\forall_{n, \hat{n}} \; n \ne \hat{n} \; d(x_n, x_{\hat{n}}) \ge \varepsilon$. \\
    \textbf{zdystansowany} $\iff \exists_{\varepsilon > 0}$ zdystansowany o $\varepsilon$.
    \end{quote}

    \subsubsection*{Twierdzenie o kryterium całkowitej ograniczoności}
    \begin{quote}
    Zbiór jest \textbf{całkowicie ograniczony} $\iff$ nie istnieje w nim ciąg zdystansowany.
    \end{quote}
}
\zagadnienie{AM3-W5, 16.10.25 (bez uzupełnionych dowodów)}{}
{
\subsubsection*{Definicja punktu skupienia ciągu}
    \begin{quote}
    $(X, \mathfrak{T})$, $(x_n)$, $x_n \in X$. \\
    Powiemu, że $x_0 \in X$ jest \textbf{punktem skupienia ciągu $(x_n)$} $\iff$ \\
    $x_0 \in \mathrm{Acc}(x_n)_{n=1}^\infty = \{x \in X: \forall_{x \in U \in \mathfrak{T}} \; \#\{n \in \mathbb{N} : x_n \in U\} = \infty\}$ \\
    Uwaga: $\mathrm{Lim}(x_n) \subset \mathrm{Acc}(x_n)$
    \end{quote}

    \subsubsection*{Twierdzenie (o warunkach zwartości w przestrzeni metrycznej)}
    \begin{quote}
    $(X, d)$, $K \subset X$. NWSR: \\
    a) $K$ jest zwarty \\
    b) Każdy ciąg $(x_n)$ o wyrazach z $K$ ma punkt skupienia należący do $K$. \\
    c) Każdy ciąg $(x_n)$ o wyrazach z $K$ ma podciąg zbieżny do granicy w $K$. \\
    d) $K$ jest całkowicie ograniczony i zupełny jako przestrzeń użytkowa
    \end{quote}

    \subsubsection*{Twierdzenie o całkowitej ograniczoności w przestrzeniach $\mathbb{R}^m$}
    \begin{quote}
    Jeśli $K \subset \mathbb{R}^m$ jest ograniczony, to jest całkowicie ograniczony.
    \end{quote}

    \subsubsection*{Twierdzenie Heinego-Borela}
    \begin{quote}
    Zbiór $K \subset \mathbb{R}^m$ jest \textbf{zwarty} $\iff$ \textbf{ograniczony} i \textbf{domknięty}
    \end{quote}

    \subsubsection*{Definicja cegły w $\mathbb{R}^n$}
    \begin{quote}
    $C = \prod_{i=1}^n [c_i, \hat{c}_i]$, $c_i \le \hat{c}_i$
    \end{quote}

    \subsubsection*{Definicja objętości}
    \begin{quote}
    Objętością cegły nazywamy liczbę:
    $\mathrm{Vol}(C) = \prod_{i=1}^n (\hat{c}_i - c_i)$
    
\end{quote}
    \subsubsection*{Lemat (szkolny)}
    \begin{quote}
    Zachodzi równość \\
    $(a_{1,1} + \dots + a_{1, p_1})(a_{2,1} + \dots + a_{2, p_2})\dots(a_{m,1} + \dots + a_{m, p_m}) = \prod_{j=1}^m \sum_{k=1}^{p_j} a_{j,k} = \sum_{\varphi \in \prod_{j=1}^m \{1, \dots, p_j\}} \prod_{j=1}^m a_{j, \varphi(j)}$ \\
    (rozdzielność mnożenia względem dodawania).
    \end{quote}

}
\zagadnienie{AM3-W6, 22.10.25}{}
{
}
\zagadnienie{AM3-W7, 22.10.25}{}
{
}

\end{document}
