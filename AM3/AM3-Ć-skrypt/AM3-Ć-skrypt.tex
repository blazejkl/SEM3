\documentclass[10pt,a4paper]{article}

% Pakiety
\usepackage[utf8]{inputenc} % Kodowanie znaków
\usepackage[T1]{fontenc} % Poprawne wyświetlanie polskich znaków
\usepackage[polish]{babel} % Ustawienie języka polskiego
\usepackage{geometry} % Ustawienia strony
\usepackage{fancyhdr} % Nagłówki i stopki
\usepackage{titlesec} % Formatowanie sekcji
\usepackage{amsmath} % Dodatkowe komendy matematyczne
\usepackage{enumitem}
\usepackage{amssymb} % Symbole matematyczne, w tym \mathbb
\usepackage{multicol}
\usepackage[most]{tcolorbox}
\tcbset{
  mybox/.style={
    colback=gray!5, % kolor tła
    colframe=black, % kolor ramki
    boxrule=0.5pt,  % grubość ramki
    arc=2pt,        % zaokrąglenie rogów
    left=6pt,right=6pt,top=6pt,bottom=6pt
  }
}

% Ustawienia strony
\geometry{margin=2cm}

% Nagłówek
\pagestyle{fancy}
\fancyhf{} % Czyszczenie nagłówków i stopek
\fancyhead[C]{AM3-Ć-skrypt}
\fancyhead[R]{\thepage} % numer strony po prawej
\renewcommand{\headrulewidth}{0pt} % Usunięcie linii w nagłówku


% Sekcja tytułowa na każdej stronie
\newcommand{\zagadnienie}[3]{%
    \clearpage % Każde zagadnienie na nowej stronie
    \noindent\textbf{#1} #2\\
    #3
}

% Dokument
\begin{document}

% treść wykładu
\zagadnienie{AM3-Ć1, 03.10.25}{}
{
    \begin{tcolorbox}[mybox]
    \subsubsection*{Przestrzzeń topologiczna}
    \subsubsection*{Pytanie 1.}
    \end{tcolorbox}
    \subsubsection*{Zadanie 1.}
    \begin{tcolorbox}[mybox]
    \subsubsection*{Uwaga}
    \subsubsection*{Przestrzeń metryczna}
    \subsubsection*{Baza topologii}
    \end{tcolorbox}
    \subsubsection*{Zadanie 2.}
}
\zagadnienie{AM3-Ć2, 07.10.25}{}
{
    \begin{tcolorbox}[mybox]
    \subsubsection*{Wnętrze}
    \subsubsection*{Domknięcie}
    \end{tcolorbox}
    \subsubsection*{Zadanie 3.}
    \subsubsection*{Zadanie 4.}
    \begin{tcolorbox}[mybox]
    \subsubsection*{Brzeg}
    \subsubsection*{Tw. o trójpodziale}
    \subsubsection*{Przykład w $\mathbb{R}$}
    \subsubsection*{Topologia indukowana}
    \subsubsection*{Zbiory graniczne}
    \subsubsection*{Przykłady}
    \subsubsection*{Przestrzeń Hausdorffa}
    \end{tcolorbox}
}
\zagadnienie{AM3-Ć3, 10.10.25}{}
{
    \subsubsection*{Zadanie 5.}
    \subsubsection*{Zadanie 6.}
    \subsubsection*{Zadanie 7.}
    \subsubsection*{Zadanie 8.}
    \subsubsection*{Zadanie 9.}
}
\zagadnienie{AM3-Ć4, 14.10.25}{}
{
    \subsubsection*{Zadanie 10.}
    \subsubsection*{Zadanie 11.}
    \subsubsection*{Zadanie 12.}
    \subsubsection*{Zadanie 13.}
    \subsubsection*{Zadanie 14.}
}
\zagadnienie{AM3-Ć5, 17.10.25}{}
{
    \subsubsection*{Zadanie 15.}
    \subsubsection*{Zadanie 16.}
    \subsubsection*{Zadanie 17.}
    \subsubsection*{Zadanie 18.}
}
\zagadnienie{AM3-Ć6, 21.10.25}{}
{
}
\zagadnienie{AM3-Ć7, 24.10.25}{}
{

}



\end{document}
